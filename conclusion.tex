\section*{Заключение}
\addcontentsline{toc}{section}{Заключение}

В ближайшем будущем проблема плагиата будет только набирать обороты в связи с всё более широким распространением сети Интернет. Поэтому необходимо уже сейчас разрабатывать меры, которые позволят в дальнейшем не усугубить эту проблему. Одной из таких мер может выступать автоматизация процесса проверки работ на плагиат.

В данной работе было успешно показано, как можно спроектировать и реализовать прототип системы, которая в совокупности с использованием платформы дистанционного обучения позволяет автоматизировать процесс проверки работ на плагиат. После внедрения такой ИС сотруднику учебного заведения достаточно будет пронализировать результаты проверки и принять решение.

Так же стоит заметить, что разработанная ИС является прототипом, который необходимо дорабтать, что бы полностью раскрыть потенциал автоматической проверки работ с помощью информационных технологий. В качестве первостепенных задач для доработки можно выделить разработку интеграционного плагина для ИС <<Moodle>>, добавление поддержки новых языков для проверки и добавление новых алгоритмов для выявления схожести работ.