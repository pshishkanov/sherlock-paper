\documentclass[russian, utf8]{article}
% Кодировка шрифтов в pdf
\usepackage{cmap} 

\pagestyle{empty}

% Настройка шрифтов
\usepackage[T2A]{fontenc}
\usepackage[utf8]{inputenc}
\usepackage[russian]{babel}

 % Полуторный интервал
\linespread{1.4}
\begin{document}
\begin{center}Аннотация\end{center}

Темой данной выпускной работы является проектирование и разработка ИС для проверки студенческих работ на плагиат с возможностью интеграции этой ИС с уже существующей платформой дистанционного обучения с целью частичной автоматизации процесса проверки работ.

Этап проверки на плагиат студенческих работ должен быть неотъемлемой частью общего процесса проверки работ, так как черезмерное заимствование студентами чужих работ приводит к снижению общего уровня знаний. Готовые решения не подходят учебным заведениям ввиду высокой платы за использование или отсутствие возможности интеграции с уже используемыми системами в учебном заведении. 

Основное преимущество разрабатываемой системы - это возможность размещения системы проверки на плагиат в рамках учебного заведения и возможность быстрого добавления необходимого функционала.

Работа представленна на 58 листах, включая 17 таблиц, 10 рисунков, 17 формул. \\

The theme of this work is the design and development of information system to check student work for plagiarism with the ability to integrate this information system to the existing e-learning platform for the purpose of partial automation of the verification process works.

Stage check for plagiarism student work should be an integral part of the overall verification process works, as excessive borrowing by students of the work of others leads to reduction of the General level of knowledge. At the moment there are already a number of solutions that provide a service to check papers for plagiarism. But sometimes these solutions are not suitable for educational institutions due to high price for the use or the lack of integration with existing systems already in use in the school. 

The main advantage of the developed system is the ability to deploy the system checks for plagiarism within the institution and the ability to quickly add the required functionality.

The work presented on 58 leaves, including 17 tables, 10 figures, 17 formulas.
\end{document}