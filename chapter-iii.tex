\section{Реализация прототипа ИС}
	
	\subsection{Cостав и структура команды разработки}

		Для разработки ИС была укомплектована команда, состоящя из следующих участиков:
		\begin{itemize}			
			\item прикладной математик;
			\item проектировщик;
			\item разработчик (backend);
			\item разработчик (fronend);
			\item тестировщик.
		\end{itemize}

		Задача прикладного математикак - реализация необходимых алгоритмов для выявления совпадений в документах.

		Задача проектировщика - разработать архитектуру будущей ИС и выбор технологий для реализации ИС.

		Задача разработчика (backend) - реализация ИС с учётом выбранной архитектуры и технологий, а так же с использованием наработок математика.

		Задача разработчика (fronend) - реализация плагина для ИС <<Moodle>>, который предоставляет интерфейс для работы с ИС <<Шерлок>>.

		Задача тестировщика - проверка корректности работы ИС и проверка на соответствие с требованиями.

	\subsection{Выбор и описание средств владения кода}

		В качестве системы управления версиями исходного кода был выбран Git - эта система получшила широкое распространение и интеграцию в сторонние продукты, содержит необходимый функционал, а так же бесплатна. 

	\subsection{Алгоритмы, использумые в разработке ИС.}

		Для выявления схожести между документами применяется несколько алгоритмов, но не один из них не является лучшим для проверки разных типов работ. В текущей версии ИС <<Шерлок>> используются только один алгоритм: RKR-GST (Running-Karp-Rabin Greedy-String-Tiling) \cite{Wise1993}. В дальнейшем будет добавлен алгоритм Winnowing \cite{Winnowing2003}:

		\subsubsection{Описание алгоритма RKR-GST}

			Данный алгоритм состоит из двух этапов. На первом этапе происходит поиск наиболее длинной последовательности в обеих документах. Для поиска совпадений используется алгоритм Рабина — Карпа \cite{Karp1987} и длина совпадения должна быть больше минимального значения, заданного изначально. На втором этапе совпадения помечаются, что бы в дальнейшем уже найденные совпадения не использовались на первом этапе при последующих итерациях. После того, как все совпадения будут помечены, первый этап алгоритма начинается снова. Алгоритм завершается в том случае, когда длина найденного совпадения меньше минимального значения, заданного изначально. В приложении Б представлен псевдокод данного алгоритма.

	\newpage
	\subsection{Этапы обработки запроса}

		Процесс обработки запроса на проверку на плагиат состоит из нескольких этапов:
		\begin{itemize}			
			\item препроцессинг;
			\item токенизация;
			\item вычисление схожести;
			\item вычисление итогового результата.
		\end{itemize}

		Качество реализации этапов препроцессинга и токенизации играет большое влияние на итоговый результат проверки на схожесть работ \cite{Kleiman2009}. 

		\subsubsection{Препроцессинг}

			На данном этапе происходит ``очистка'' проверяемой работы от таких изменений, как удаление комментариев, разделение или слияние блоков определения переменных, изменения порядка определения переменных, добавление лишних операторов. Такие изменения вносятся в исходную работу студентом с целью снижения вероятности обнаружения плагиата в работе.

		\subsubsection{Токенизация}

			На данном этапе происходит преобразование проверяемой работы в набор токенов. Например, все значения и переменные заменяются на токены <IDENT> и <VALUE> соответственно. Данный процесс позволяет преобразовать работу в такой формат, с котором наиболее эффективно работают алгоритмы сравнения. В данной работе для токенизации используется библиотека ANTLR \cite{Parr2013}.

		\subsubsection{Вычисление схожести}

			На данном этапе происходит сравнение работы из запроса с работами, которые хранятся в БД. Для сравнения отбираются не все работы, а только те, которые имеют схожую тематику с проверяемой работой. При сравнении вычисляется значение схожести с использованием каждого алгоритма по отдельности.

			Для алгоритма RKR-GST значение схожести между работами определяется по формуле \ref{eq:sim}.
			\begin{equation}\label{eq:sim}
				sim(a,b) = \frac{ 2 * coverage }{ length(a) + length(b) },
			\end{equation}
			\begin{ESKDexplanation}
				\item[где ] $length(a)$ --- число токенов в работе $a$;
				\item$length(b)$ --- число токенов в работе $b$;
				\item$coverage$ --- число совпадений в работах.		
			\end{ESKDexplanation}

		\subsubsection{Вычисление итогового результата}		

			Итоговый результат проверки опрелеяется по формуле \ref{eq:result_sim}.

			\begin{equation}\label{eq:result_sim}
				sim(a,b) = \sum_{i=1}^{n} \frac{ w_i * sim_i(a,b) }{ w },				 
			\end{equation}
			\begin{ESKDexplanation}
				\item[где ] $sim_i$ --- рассчитанное значение схожести между работами для алгоритма i;
				\item$w_i$ --- вес алгоритма;
				\item$w$ --- общий вес всех алгоритмов;
				\item$n$ --- число используемых алгоритмов при проверке.		
			\end{ESKDexplanation}


