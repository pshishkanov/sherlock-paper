\section*{Введение}
\addcontentsline{toc}{section}{Введение}

	С давних времён люди выражают свои мысли разными способами, но при этом смысл высказываний может быть один и тот же. В большинстве ситуаций такое поведение допустимо, но бывают случаи, когда происходит умышленное присвоение авторства той или иной работы, то есть имеет место плагиат \cite{Plagiat2002}. Степень плагиата можно оценивать от плагиата <<в чистом виде>>, когда изначальная работа вообще не перерабатывается, до полной переработки исходной работы, когда от работы остаётся только общий смысл.

	Также как и в других сферах деятельности, в образовательной среде остро стоит проблема плагиата. Студенты могут обмнениваться между собой своими работами, что может повлиять как на корректность оценки знаний студентов, так и на уровень образования в целом. Ещё острее эта проблема стала после повсеместного распространения Интернета - в сети доступны множество различных ресурсов, на котороых можно найти множество готовых работ. Всё это затрудняет работу преподавателя, так как ему необходимо проверять работы не только на корректность выполнения, но и на наличие плагиата.

	Наряду с недостатками, распространение Интернета в сфере обучения принесло множество новых возможностей. Одной из таких возможностей является модель дистанционого обучения. При таком подходе преподаватель и студент взаимодействуют друг с другом через платформу дистанционного обучения, которая доступна из любого места, где есть Интернет. Одной из таких платформ является система <<Moodle>>.

	Использование платформы дистанционного обучения в совокупности с применением компьютерных методов поиска и обнаружения плагиата позволяет автоматизировать этап проверки работ на заимствования, что влечёт за собой сокращение времени проверки работ студентов и повышение качества проверки на плагиат.